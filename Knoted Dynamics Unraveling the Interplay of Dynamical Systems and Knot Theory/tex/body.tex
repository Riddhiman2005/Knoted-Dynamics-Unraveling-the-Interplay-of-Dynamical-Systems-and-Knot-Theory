\documentclass[main.tex]{subfiles}

\begin{document}


\section{Proof that $ \mathcal{V}$ is a universal template}

In this section, I'll outline the proof given in full detail in \cite{Ghrist1996} that $\V$ (shown in Figure~\ref{fig:universal}) is a universal template. Details will inevitably be omitted but our goal in this exposition is to straddle the line between
the full proof and the very high-level outline given in~\cite{knottyode}, giving a concise and readable exposition that gets at the essence of the proof.

The essential idea is to find a special set of templates that have a special property that forces them to support all braids as orbits, and then to show that these templates can be found in $\V$. Since all knots and links
can be realized as a braid, this immediately implies that $\V$ is universal. The last step of showing that the special braid-supporting templates can found in $\V$ is the most difficult by far and where we will have to do the
most hand-waving. Nonetheless, I'll try to capture the elegance and the beauty.

Also important in this proof is the existence of the $\mathcal{U}$ template, pictured below in Figure~\ref{fig:universal}. $\mathcal{U}$ and $\V$ are essentially equivalent, with $x_3$ and $x_4$ swapped. $\mathcal{U}$ also universal.




\subsection{Braids and the theorem of Alexander}

Recall that a closed braid is a collection of $P$ disjoint simple closed curves in a standardly embedded torus $D^2 \times S^1$ such that every $D^2$ cross-section of the torus intersects the closed braid in exactly $P$ points.

% related to pure braid group/symmetric group -- this following line isn't entirely accurate, so I've commented it out for now
%Braids have a natural identification as a permutation on $P$ elements, which is the property we exploit.

In a landmark paper~\cite{Alexander1923}, Alexander proved the following theorem which provides the crucial connection
between braids and links.

% we exploit it since each permutation can be written as the product of exchanges

\begin{thm}[Alexander 1923]
  Each knot or link is isotopic to some closed braid on $P$ strands for some $P$
\end{thm}

\subsection{The templates $\W_q$}

The family of templates $\set{\W_q}$ shown in Figure~\ref{fig:w_q} are those referred to earlier as the braid-supporting templates. $\W_1$ is identically $\V$ and increasing $q$ by one has the effect of adding
two \emph{ears} of alternating `sign' to one side. The property that successive ears alternate in `sign' is what makes this family of templates so useful, as this property makes proving that they support all braids delightfully simple.

\begin{figure}[h]
  \centering
  \includegraphics[width=0.8\textwidth]{w_q.png}
  \caption{The templates $W_q$ \cite{knottyode}}\label{fig:w_q}
\end{figure}


\begin{lemma}[Ghrist 1996]
  An isotopic copy of any closed braid exists as a set of periodic orbits on some $W_q$ for sufficently large $q$.
\end{lemma}

\begin{proof}

    The concatenation of alternating positive and negative ears on $\W_q$ mimics the braid group concatenation operation. The braid group $B_{n+1}$ is naturally generated by $\sigma_1, \sigma_2, \dots, \sigma_{n}$, where $\sigma_i$ is a single crossing of the $i$th strand over the $(i + 1)$th strand and its inverse, $\sigma_i^{-1}$, is a single crossing of the $(i + 1)$th strand over the $i$th strand. Naturally, $\sigma_i \sigma_i^{-1} = \sigma_i^{-1} \sigma_i = I$. We can consider the elements $\pi_1, \pi_2, \dots, \pi_n, \pi_1', \pi_2', \dots, \pi_n'$ of $B_{n+1}$, where $\pi_i = \sigma_1 \sigma_2 \cdots \sigma_i$ and $\pi_i' = \sigma_1^{-1} \sigma_2^{-1} \cdots \sigma_i^{-1}$. Note that $\pi_1 = \sigma_1$, $\pi_1' = \sigma_1^{-1}$, and that for $i > 1$,

    \begin{align*}
        \pi_{i-1}^{-1} \pi_i  &= (\sigma_1 \sigma_2 \dots \sigma_{i - 1})^{-1} \sigma_1 \sigma_2 \cdots \sigma_i \\
                              &= \sigma_{i - 1}^{-1} \cdots \sigma_2^{-1} \sigma_1^{-1} \sigma_1 \sigma_2 \cdots \sigma_i \\
                              &= \sigma_{i - 1}^{-1} \cdots \sigma_2^{-1} \sigma_2 \cdots \sigma_i \\
                              &= \sigma_{i - 1}^{-1} \sigma_{i - 1} \sigma_i  \\
                              &= \sigma_i \\ \\
        \pi_{i-1}'^{-1} \pi_i'  &= (\sigma_1^{-1} \sigma_2^{-1} \dots \sigma_{i - 1}^{-1})^{-1} \sigma_1^{-1} \sigma_2^{-1} \cdots \sigma_i^{-1} \\
                                &= \sigma_{i - 1} \cdots \sigma_2 \sigma_1 \sigma_1^{-1} \sigma_2^{-1} \cdots \sigma_i^{-1} \\
                                &= \sigma_{i - 1} \cdots \sigma_2 \sigma_2^{-1} \cdots \sigma_i^{-1} \\
                                &= \sigma_{i - 1} \sigma_{i - 1}^{-1} \sigma_i^{-1}  \\
                                &= \sigma_i^{-1} \\ \\
    \end{align*}

    We can conclude that the elements $\pi_1, \pi_2, \dots, \pi_n, \pi_1', \pi_2', \dots, \pi_n'$ of $B_{n+1}$ are a generating set for $B_{n+1}$.

    %it's possible to split this proof into two, one half for proving the above is a generating set for B_{n+1} and second half for concluding that W_q contains all braids for large q

    For the braid group $B_{n+1}$, notice that for all $1 \leq i \leq n$, $\pi_i$ can be drawn on any positive ear of $\W_q$, letting all but the bottom strand take the top strip and forcing the bottom strand to take the bottom strip, cross over exactly $i$ strands, and take the top strip. Similarly, for all $i \leq i \leq n$, $\pi_i'$ can be drawn on any negative ear of $W_q$, letting all but the bottom strand take the top strip, forcing the bottom strand to take the bottom strip, cross under exactly $i$ strands, and take the top strip. Examples of $\pi_2$ and $\pi_2'$ of the braid group $B_5$ are drawn in Figure~\ref{fig:earexchange}.

    Because this set of elements generates the braid group $B_{n+1}$, any braid $B \in B_{n+1}$ can be drawn as a sequence of $\pi_i$ and $\pi_i'$. If $B$ can be represented with $m$ of these elements, then the closure of $B$ lies in $W_m$, as $W_m$ contains $m$ pairs of positive/negative ears that can each support a $\pi_i$ or $\pi_i'$ structure. Thus, any closed braid on $N$ strands lies in $W_q$ for sufficiently large $q$.

\end{proof}



\begin{figure}[h]
  \centering
  \includegraphics[width=0.8\textwidth]{ear_exchange.png}
  \caption{From Figure ~\cite{Ghrist1996}}\label{fig:earexchange}
\end{figure}





\subsection{Find $\W_q \subset \V$ for all $q$}

The remainder of the proof is to show with a careful sequence of isotopic renormalizations that $W_q$ is contained in $\V$ for all $q$. First we define the needed renormalizations.


\subsubsection{Special Inflations: $\mathcal{F}$ and $\mathcal{G}$}

In his proof, Ghrist uses a few special inflations, $\mathcal{F}$ and $\mathcal{G}$, and accompanies them with a proposition:

\begin{prop}
    The following inflations are isotopic.
$$F: U \hookrightarrow \V \begin{cases} x_1 \mapsto x_1 \\ x_2 \mapsto x_1 x_2 x_3 \\ x_3 \mapsto x_4 x_2 \\ x_4 \mapsto x_4 \end{cases}$$

$$G: \V \hookrightarrow U \begin{cases} x_1 \mapsto x_1 \\ x_2 \mapsto x_1 \\ x_3 \mapsto x_2 x_4 \\ x_4 \mapsto x_2 x_3 x_4 \end{cases}$$
\end{prop}

Diagrams witness to the fact these are valid isotopic inflations are given in Figures~\ref{fig:isotopicF} and~\ref{fig:isotopicG}. We couldn't infer isotopy from the diagram given in Figure~\ref{fig:lorenz_subtemplate}
because the subtemplate in that figure had a twist in a strip, but we can infer isotopy from the diagrams in Figures~\ref{fig:isotopicF} and~\ref{fig:isotopicG}.


\begin{figure}[h]
  \centering
  \includegraphics[width=0.8\textwidth]{inflationF.png}
  \caption{This shows that $\mathcal{F}$ is isotopic by showing that an image of $\mathcal{F}$ is isotopic to $\mathcal{U}$. Figure ~\cite{ghs1997}}\label{fig:isotopicF}
\end{figure}

\begin{figure}[h]
  \centering
  \includegraphics[width=0.8\textwidth]{inflationG.png}
  \caption{This shows that $\mathcal{G}$ is isotopic by showing that an image of $\mathcal{G}$ is isotopic to $\mathcal{V}$.  Figure ~\cite{ghs1997}}\label{fig:isotopicG}
\end{figure}


Additionally, we will need the following (\emph{not} isotopic) inflation which takes each orbit to its mirror image: $$\chi: \begin{matrix} U \mapsto U \\ \V \mapsto \V \end{matrix} \begin{cases} x_1 \mapsto x_3 \\ x_2 \mapsto x_4 \\ x_3 \mapsto x_1 \\ x_4 \mapsto x_2 \end{cases}$$

    Although $\chi$ is not isotopic on its own, its conjugation with another inflation is isotopic as seen in the following lemma.

    \begin{lemma}[Ghrist 1997]
        Given an isotopic inflation $A$ having either $\mathcal{U}$ or $\V$ as a domain and range (independently of each other), the inflation $A^* = \chi A \chi$ is also isotopic.
    \end{lemma}

    \begin{proof}
      Reidemeister moves commute with the mirror operation. That is, if $K$ is a projection of a knot and $R$ is a Reidemeister move, then $R(K^\ast)  \cong  {(R(K))}^\ast$. So although $\chi$ does not commute with $A$ as far as the symbolic
      action the inflations have on orbits, their \emph{topological} actions commute. Hence, topologically, $A^\ast$ acts as $\chi^2 A$. $\chi^2$ is the identity so is obviously isotopic. Therefore $A^\ast$ is isotopic to $A$.
    \end{proof}



The symbolic action of $\mathcal{F}^\ast$ is easily computed. For example, given $\mathcal{F}$ as defined above, $\mathcal{F}^*$ is defined as $$\mathcal{F}^*: U \hookrightarrow \V \begin{cases} x_1 \mapsto x_2 x_4 \\ x_2 \mapsto x_2 \\ x_3 \mapsto x_3 \\ x_4 \mapsto x_3 x_4 x_1 \end{cases}$$ Because $\mathcal{F}$ is isotopic, we know that $\mathcal{F}^*$ must be isotopic as well; in addition, $G^*$ is also isotopic.

We finally have the tools we need to give the proof that $\W_q$ can be found in $\V$ for all $q$.

We must start with a lemma, provided by Ghrist.
\begin{lemma}[Ghrist 1997]
    Given that a subtemplate of $\V$ doesn't contain $x_1^\infty$, it is possible to add an ear near the top branch line, like that in Figure~\ref{fig:appendToV}, which is a positive ear. Similarly, given that a subtemplate of $\V$ doesn't contain $X_3^\infty$, it is possible to add an ear near the bottom branch line, which is a negative ear. Note that this is not an inflation; we are simply adding an ear to the subtemplate. 
\end{lemma}


Now, we prove that $\W_q \subset \V$ for all $q$.

\begin{proof}
    Let us begin with $\W_1 = \V$. Let us calculate the symbolic action of the inflation $\mathfrak{H}: \V \to \V$ defined as $\mathcal{F}^* \mathcal{G} \mathcal{F} \mathcal{G}^*$. This is accomplished with the aid of Lemma \ref{lemma:symbolic} by successively expanding
letters (strips) in the space of itinerary into the admissable words they are mapped into under the inflation.
For instance, $\mathcal{G}^\ast$ takes $x_1$ to $x_4 x_2$, which when composed with $\mathcal{F}$'s action on the itineraries maps to $x_4 x_1 x_2 x_3$. Performing all of the composition in succession yields the full action:

    $$\mathfrak{H}: \V \hookrightarrow \V \begin{cases}
        x_1 \mapsto x_2 x_3^2 x_4 x_1 (x_2 x_4)^2 x_2 x_3 x_4 x_1 \\
        x_2 \mapsto x_2 x_3^2 x_4 x_1 (x_2 x_4)^3 x_2 x_3 x_4 x_1 \\
        x_3 \mapsto x_2 x_3^2 x_4 x_1  x_2 x_4  \\
        x_4 \mapsto x_2 x_3^2 x_4 x_1  x_2 x_4  \end{cases} $$

        Note that $\mathcal{G}^*$ is an isotopic inflation from $\V$ to $\mathcal{U}$, $\mathcal{F}$ is an isotopic inflation from $\mathcal{U}$ to $\V$, $\mathcal{G}$ is an isotopic inflation from $\V$ to $\mathcal{U}$, and that $\mathcal{F}^*$ is an isotopic inflation from $\mathcal{U}$ to $\V$. It must follow that $\mathfrak{H}$ is an isotopic inflation from $V$ to $V$; in other words, an isotopic renormalization.

        This renormalization satisfies the condition that its image (which is a subtemplate of $\V$) does not contain $x_1^\infty$. So, after performing this renormalization, we will be able to add a positive ear at the top branch line
        yielding an object we refer to as $\W_1^+$. To obtain $\W_2$ from this, we must add the negative ear.

        There exists a similar inflation $\mathfrak{H}^*$ from $\V$ onto $\V$ that allows for the addition of a negative ear at the bottom branch line. Now, we have added a positive ear and a negative ear, so we are at $\W_2$, yet this subtemplate is still contained within $\V$; this implies that $\W_2 \subset \V$. Now, we can repeat this procedure, adding another positive and negative ear after the respective inflations to get $\W_3$.

        This procedure can be repeated $q$ times to show that $\W_q \subset \V$ for any positive $q$; thus, by induction, $\W_q \subset \V$ for all $q$.
    \end{proof}


\begin{figure}[h]
    \centering
    \includegraphics[width=0.8\textwidth]{appendToV.png}
    \caption{Appending a positive ear to $T$ (left) yields $T^+$ (right). The thin line on (a) is a section of the subtemplate after performing $\mathcal{H}$; to add the positive ear, we widen the subtemplate so that when it splits, one half of the split returns to itself to make an infinite loop, while the other half continues as in the original subtemplate.\cite{Ghrist1996}} \label{fig:appendToV} 
    
\end{figure}



\subsection{Universal Templates and V}

\begin{thm}[Ghrist 1996]
    The template $\V$ contains an isotopic copy of every tame knot and link as a periodic orbit of the semiflow.
\end{thm}
\begin{figure}[h]
  % not sure where to put this, but not here

    %this doesn't seem like too bad of a spot; you mention figure 1 in the first paragraph of this section so it shouldn't be placed too far away from that paragraph
  \centering
  \includegraphics[width=0.8\textwidth]{universalVU.png}
  \caption{(a) The universal template $\V$ and (b) The universal template $\mathcal{U}$ \cite{ghs1997}}\label{fig:universal}
\end{figure}

\begin{proof}
    This proof comes immediately following the fact that $\W_q \subset \V$ for all $q$, and that all closed braids (and thus all tame links and knots) can be contained within $\W_q$ for large enough $q$. Thus, $\V$ contains every tame knot and link.
\end{proof}


Ghrist then extends the theorem above to prove the following:

\begin{thm}[Ghrist 1996]
    The template $\V$ contains all orientable templates as subtemplates of $\V$.
\end{thm}

It must follow that if there are any other universal templates $T$, then $\V$ must contain $T$ as a subtemplate. As $T$ contains all tame links and knots as well, as it is universal, the same theorems can be applied to $T$ as were applied to $\V$, so that $T$ also contains all templates (including $\V$).



\end{document}